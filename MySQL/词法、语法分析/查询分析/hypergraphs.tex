% !TeX program = xelatex
% !TeX TXS-program:xelatex = -shell-escape

\documentclass[fontset=none]{ctexart}
\ctexset{fontset=fandol}

\usepackage{xeCJK}
\usepackage[a4paper, left=1cm, right=1cm, top=1cm, bottom=1.5cm]{geometry}    %设置页面边距
\usepackage{graphicx}   %插入图片

\usepackage{minted}     %允许插入代码段。需要在settings.json中"latex-workshop.latex.tools"节点的子节点"name":"xelatex"下面添加"-shell-escape",
% \usepackage{minted} % 引用minted包
\usepackage{caption} % 引用caption包
\usepackage{fvextra} % 引用fvextra包
\DeclareCaptionType{code}[Code Listing][List of Code Listings] % 设置code环境为新的代码环境


%\setCJKmainfont{STHeiti} % 设置字体
%\setCJKsansfont{STHeiti}
%\setCJKmonofont{STHeiti}

\usepackage{amsmath} % 数学公式包
\usepackage{amssymb} % 实数集的包
\usepackage{parskip} % 段落包
\usepackage{setspace} % 行距包
% \setlength{\parindent}{0pt} % 取消段落缩进
\usepackage{titlesec} % 章节标题格式包
\titleformat*{\section}{\large\bfseries} % 设定section标题格式为加粗大号字体
\usepackage{tocloft} % 目录格式包
\renewcommand\cftsecfont{\normalfont\bfseries} % 设定section标题格式为加粗字体
\renewcommand\cftsecpagefont{\normalfont\bfseries} % 目录页码字体
\usepackage{ragged2e} % 左对齐包
\usepackage{hyperref} % 超链接包
\usepackage{array} % 引入 array、表格 包
% \usepackage{amsthm} % 使用definition环境
\hypersetup{colorlinks=true, linkcolor=blue, filecolor=magenta, urlcolor=blue} % 超链接格式设置


\title{超图学习}
\author{Lindeci}
\date{\today}

\begin{document}
\maketitle
\tableofcontents

\definecolor{codebgcolor}{HTML}{F8F8F8} % 设置背景颜色

\section{概念}

\newtheorem{definition}{定义}

\begin{definition}{(Hypergraph)}
    超图是一对普通点集合和超边集合 $H = (V,E)$ such that
    \begin{enumerate}
    \item $V$ 是普通点的集合
    \item $E$ 是超边的集合。
    
    超边连接着一对无序的点集合 $(u, v)$ ,这对点集合满足 
    
    $u \subset V$ 且 $v \subset V$ 且 $u \cap v = \emptyset$
    \end{enumerate}

    \ \ \ $V$ 的非空子集叫做超节点(hypernode)。

    \ \ \ $V$ 中的所有普通节点的某种排序,我们称为\ 关系$\prec$\ 。

    \ \ \ 在我们的语境中,$V$ 中的所有普通节点是表的抽象,超边是连接谓词。

%    \ \ \ 考虑一个形如 $R1.a + R2.b + R3.c = R4.d + R5.e + R6.f$ 的联接谓词。这个谓词将产生一个超边 $(\{R1, R2, R3\}, \{R4, R5, R6\})$。
    $V$ 的普通节点集是 $V = \{R1, \ldots , R6\}$。关于普通节点排序,我们假设 $Ri \prec Rj \Leftrightarrow i < j$。
    有简单的边 $(\{R1\},\{R2\})$,$(\{R2\},\{R3\})$,$(\{R4\},\{R5\})$和 $(\{R5\}, \{R6\})$。
\end{definition}

\begin{definition}{(subgraph)}
    设 H = (V, E) 是一个超图,$V' \subseteq V$, $V'$ 是普通节点的一个子集。

    \ \ \ $G|V'$ 定义为 $G|V' = (V',E')$,其中 $E' = \{(u,v)|(u,v) \in E,u \subseteq V',v \subseteq V'\}$。
    
    \ \ \ $V'$ 上的节点排序继承 $V$ 上节点排序。
\end{definition}

\begin{definition}{(connected)}
%\textbf{定义 3 (连通)}。设 H = (V, E) 是一个超图。当 |V| = 1 或存在一个 V 的划分 V',V'' 和一个超边 (u, v) ∈ E,使得 u ⊆ V',v ⊆ V'',且 G|V' 和 G|V'' 都是连通的,那么 H 是连通的。
\end{definition}

\end{document}

