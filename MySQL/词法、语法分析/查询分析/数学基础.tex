% !TeX program = xelatex
% !TeX TXS-program:xelatex = -shell-escape

\documentclass[fontset=none]{ctexart}
\ctexset{fontset=fandol}

\usepackage{xeCJK}
\usepackage[a4paper, left=1cm, right=1cm, top=1cm, bottom=1.5cm]{geometry}    %设置页面边距
\usepackage{graphicx}   %插入图片

\usepackage{minted}     %允许插入代码段。需要在settings.json中"latex-workshop.latex.tools"节点的子节点"name":"xelatex"下面添加"-shell-escape",
% \usepackage{minted} % 引用minted包
\usepackage{caption} % 引用caption包
\usepackage{fvextra} % 引用fvextra包
\DeclareCaptionType{code}[Code Listing][List of Code Listings] % 设置code环境为新的代码环境

\usepackage{amsmath} % 数学公式包
\usepackage{amssymb} % 实数集的包
\usepackage{parskip} % 段落包
\usepackage{setspace} % 行距包
% \setlength{\parindent}{0pt} % 取消段落缩进
\usepackage{titlesec} % 章节标题格式包
\titleformat*{\section}{\large\bfseries} % 设定section标题格式为加粗大号字体
\usepackage{tocloft} % 目录格式包
\renewcommand\cftsecfont{\normalfont\bfseries} % 设定section标题格式为加粗字体
\renewcommand\cftsecpagefont{\normalfont\bfseries} % 目录页码字体
\usepackage{ragged2e} % 左对齐包
\usepackage{hyperref} % 超链接包
\usepackage{array} % 引入 array、表格 包
% \usepackage{amsthm} % 使用definition环境
\hypersetup{colorlinks=true, linkcolor=blue, filecolor=magenta, urlcolor=blue} % 超链接格式设置


\title{超图学习}
\author{Lindeci}
\date{\today}

\begin{document}
\maketitle
\tableofcontents

\definecolor{codebgcolor}{HTML}{F8F8F8} % 设置背景颜色

\section{概念}

\newtheorem{definition}{定义}


\begin{definition}{(subgraph)}
    设 H = (V, E) 是一个超图,$V' \subseteq V$, $V'$ 是普通节点的一个子集。

    \ \ \ $G|_{V'}$ 定义为 $G|_{V'} = (V',E')$,其中 $E' = \{(u,v)|(u,v) \in E,u \subseteq V',v \subseteq V'\}$。
    
    \ \ \ $V'$ 上的节点排序继承 $V$ 上节点排序。

    $\Join $
 
    ⨝
    ⨝ nature join
    ⟗ full outerjoi
    ⟕ left outerjoin
    ⟖ right outerjoin
    ✕ cross join
    ⋊ 
    ⋉ 
    ⋉
    ▷

\end{definition}


\end{document}

